\documentclass[12pt]{article}
\usepackage{amsmath}
\usepackage{amsfonts}
\usepackage{amssymb}
\usepackage{geometry}
\usepackage{xcolor}
\geometry{a4paper}

% % Define color theme
% \definecolor{factorA}{RGB}{119,221,119} % Pastel Hunter Green
% \definecolor{factorB}{RGB}{204,255,204} % Pastel Lime Green
% \definecolor{operator}{RGB}{255,179,71} % Pastel Orange
% \definecolor{alternate}{RGB}{150,111,214} % Pastel Dark Purple

% Define color theme
\definecolor{factorA}{RGB}{119,221,119} % Pastel Hunter Green
\definecolor{factorB}{RGB}{255,179,71} % Pastel Orange
\definecolor{operator}{RGB}{211,85,85} % Pastel Maroon
\definecolor{alternate}{RGB}{150,111,214} % Pastel Dark Purple

\title{Algebraic Simplification Rules}
\author{Jérémie O. Lumbroso}
\date{August 26, 2023}

\begin{document}
\maketitle

\section*{1. Combining Like Terms}
When you have terms that look the same, you can add or subtract them together.
\begin{align*}
    {\color{factorA}3}x {\color{operator}+} {\color{factorA}4}x & = ({\color{factorA}3 {\color{operator}+} 4})x = {\color{factorA}7}x \\
    {\color{factorA}5}y {\color{operator}-} {\color{factorA}2}y & = ({\color{factorA}5 {\color{operator}-} 2})y = {\color{factorA}3}y \\
    {\color{factorA}6}a {\color{operator}+} {\color{factorA}2}a {\color{operator}-} {\color{factorA}4}a & = ({\color{factorA}6 {\color{operator}+} 2 {\color{operator}-} 4})a = {\color{factorA}4}a
\end{align*}
\textbf{Note:} This rule is derived from the distributive property of multiplication over addition.

\section*{2. Distributive Property}
When you multiply a number outside of a bracket with terms inside the bracket, you multiply that number with each term inside.
\begin{align*}
    3({\color{factorA}x} {\color{operator}+} {\color{factorB}y}) & = 3{\color{factorA}x} + 3{\color{factorB}y} \\
    4({\color{factorA}2z} {\color{operator}-} {\color{factorB}3}) & = 4{\color{factorA}z} - {\color{factorB}12} \\
    5({\color{factorA}a} {\color{operator}-} {\color{factorB}b} {\color{operator}+} {\color{alternate}c}) & = 5{\color{factorA}a} - 5{\color{factorB}b} + 5{\color{alternate}c}
\end{align*}
\textbf{Note:} This rule is a direct application of the distributive property of multiplication over addition: \(a(b + c) = ab + ac\).

\section*{3. Multiplying Powers with the Same Base}
When you multiply numbers with the same base and different powers, you add the powers together.
\begin{align*}
    x^{{\color{factorA}2}} \times x^{{\color{factorB}3}} & = x^{{\color{factorA}2} {\color{operator}+} {\color{factorB}3}} = x^{{\color{alternate}5}} \\
    y^{{\color{factorA}4}} \times y^{{\color{factorB}2}} & = y^{{\color{factorA}4} {\color{operator}+} {\color{factorB}2}} = y^{{\color{alternate}6}} \\
    z^{{\color{factorA}3}} \times z^{{\color{factorB}7}} & = z^{{\color{factorA}3} {\color{operator}+} {\color{factorB}7}} = z^{{\color{alternate}10}}
\end{align*}
\textbf{Note:} This rule is derived from the properties of exponents: \(a^m \times a^n = a^{m+n}\).

\section*{4. Dividing Powers with the Same Base}
When you divide numbers with the same base and different powers, you subtract the powers.
\begin{align*}
    \frac{x^{{\color{factorA}5}}}{x^{{\color{factorB}2}}} & = x^{{\color{factorA}5} {\color{operator}-} {\color{factorB}2}} = x^{{\color{alternate}3}} \\
    \frac{y^{{\color{factorA}6}}}{y^{{\color{factorB}3}}} & = y^{{\color{factorA}6} {\color{operator}-} {\color{factorB}3}} = y^{{\color{alternate}3}} \\
    \frac{z^{{\color{factorA}8}}}{z^{{\color{factorB}3}}} & = z^{{\color{factorA}8} {\color{operator}-} {\color{factorB}3}} = z^{{\color{alternate}5}}
\end{align*}
\textbf{Note:} This rule is derived from the properties of exponents: \(\frac{a^m}{a^n} = a^{m-n}\).


\section*{5. Power of a Power}
When you have a power raised to another power, you multiply the powers.
\begin{align*}
    (x^{{\color{factorA}2}})^{{\color{factorB}3}} & = x^{{\color{factorA}2} {\color{operator}\times} {\color{factorB}3}} = x^{{\color{alternate}6}} \\
    (y^{{\color{factorA}3}})^{{\color{factorB}4}} & = y^{{\color{factorA}3} {\color{operator}\times} {\color{factorB}4}} = y^{{\color{alternate}12}} \\
    (z^{{\color{factorA}4}})^{{\color{factorB}2}} & = z^{{\color{factorA}4} {\color{operator}\times} {\color{factorB}2}} = z^{{\color{alternate}8}}
\end{align*}
\textbf{Note:} This rule is derived from the properties of exponents: \((a^m)^n = a^{m \times n}\).

\section*{6. Zero Exponent Rule}
Any number (except zero) raised to the power of zero is always 1.
\begin{align*}
    x^{{\color{operator}0}} & = {\color{alternate}1} \\
    5^{{\color{operator}0}} & = {\color{alternate}1} \\
    a^{{\color{operator}0}} & = {\color{alternate}1}
\end{align*}
\textbf{Note:} This rule is derived from the properties of exponents: \(a^0 = 1\) for all \(a \not= 0\).

\section*{7. Negative Exponent Rule}
A number with a negative exponent becomes a fraction with a positive exponent in the denominator.
\begin{align*}
    x^{{\color{operator}-2}} & = \frac{1}{x^{{\color{factorA}2}}} \\
    y^{{\color{operator}-3}} & = \frac{1}{y^{{\color{factorA}3}}} \\
    2^{{\color{operator}-4}} & = \frac{1}{2^{{\color{factorA}4}}}
\end{align*}
\textbf{Note:} This rule is derived from the properties of exponents: \(a^{-n} = \frac{1}{a^n}\) for all \(a \not= 0\).

\section*{8. Factorization}
Breaking down a number or expression into its simplest parts (factors) that, when multiplied, give the original number or expression.
\begin{align*}
    12 & = {\color{factorA}3} \times {\color{factorB}4} \\
    x^2 - 9 & = (x {\color{operator}+} {\color{factorA}3})(x {\color{operator}-} {\color{factorB}3}) \\
    y^2 + 2y + 1 & = (y {\color{operator}+} {\color{factorA}1})(y {\color{operator}+} {\color{factorB}1})
\end{align*}
\textbf{Note:} This rule is derived from the properties of algebraic expressions. For instance, \(a^2 - b^2 = (a+b)(a-b)\) and \((a+b)^2 = a^2 + 2ab + b^2\).


\newpage

\section*{Practice Exercises}

\subsection*{Combining Like Terms}
\begin{enumerate}
    \item Simplify: \(2m + 5m\)
    \item Simplify: \(7n - 3n + 2n\)
    \item Simplify: \(4p - p\)
    \item Simplify: \(9q + q - 5q\)
\end{enumerate}

\subsection*{Distributive Property}
\begin{enumerate}
    \item Expand: \(2(x + 3)\)
    \item Expand: \(3(2y - 4)\)
    \item Expand: \(4(z + 2 - z)\)
    \item Expand: \(5(a - b + 2c)\)
\end{enumerate}

\subsection*{Multiplying Powers with the Same Base}
\begin{enumerate}
    \item Simplify: \(x^2 \times x^3\)
    \item Simplify: \(y^4 \times y^2\)
    \item Simplify: \(z^3 \times z^7\)
    \item Simplify: \(a^5 \times a^2\)
\end{enumerate}

\subsection*{Dividing Powers with the Same Base}
\begin{enumerate}
    \item Simplify: \(\frac{x^5}{x^2}\)
    \item Simplify: \(\frac{y^6}{y^3}\)
    \item Simplify: \(\frac{z^8}{z^3}\)
    \item Simplify: \(\frac{a^7}{a^4}\)
\end{enumerate}
\subsection*{Power of a Power}
\begin{enumerate}
    \item Simplify: \((x^2)^3\)
    \item Simplify: \((y^3)^4\)
    \item Simplify: \((z^4)^2\)
    \item Simplify: \((a^5)^2\)
\end{enumerate}

\subsection*{Zero Exponent Rule}
\begin{enumerate}
    \item Simplify: \(x^0\)
    \item Simplify: \(5^0\)
    \item Simplify: \(a^0\)
    \item Simplify: \(b^0\)
\end{enumerate}

\subsection*{Negative Exponent Rule}
\begin{enumerate}
    \item Simplify: \(x^{-2}\)
    \item Simplify: \(y^{-3}\)
    \item Simplify: \(2^{-4}\)
    \item Simplify: \(a^{-5}\)
\end{enumerate}

\subsection*{Factorization}
\begin{enumerate}
    \item Factorize: \(x^2 - 9\)
    \item Factorize: \(y^2 + 2y + 1\)
    \item Factorize: \(z^2 - 4\)
    \item Factorize: \(a^2 - 16\)
\end{enumerate}


\end{document}
